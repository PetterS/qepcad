%%%%%%%%%%%%%%%%%%%%%%%%%%%%%%%%%%%%%%%%%%%%%%%%%%%%%%%%%%%%%%%%%%%%%%%%%%%
\subsection{Implementation of Lists}

When \saclib\ is initialised, the array \SPACE\ containing $\NU+1$ \Word s is
allocated from the heap. This array is used as the memory space for list
processing. Lists are built from {\em cells}, which are pairs of
consecutive \Word s the first of which is at an odd position in the \SPACE\
array. {\em List handles} (``pointers'' to lists) are defined to be \BETA\
plus the index of the first cell of the list in the \SPACE\ array with the
handle of the empty list being \NIL\ (which is equal to \BETA). Figure
\ref{f:SPACE} shows the structure of the \SPACE\ array.

\begin{figure}[htb]
\unitlength0.5pt
\begin{picture}(640,120)
\savebox{\cell}(0,0)[t]{
  \put( 15,  8){\oval(30, 8)[bl]}
  \put( 15,  0){\oval(30, 8)[tr]}
  \put( 45,  0){\oval(30, 8)[tl]}
  \put( 45,  8){\oval(30, 8)[br]}
  \put( 30, -4){\makebox(0,0)[t]{cell}}
}

\put( 65, 0){\begin{picture}(550,100)
  \put(  0,  5){\SPACE}
  \put(  0,  0){\line(  1,0){ 50}}
  \put( 50,  0){\vector(  2,  1){ 50}}
  \put(110, 30){\framebox(210, 30){}}
  \multiput(140,30)(30,0){6}{\line(0,1){30}}
  \put(321, 30){\usebox{\dline}}
  \put(321, 60){\usebox{\dline}}
  \multiput(340, 45)(15,0){3}{\circle*{2}}
  \put(374, 30){\usebox{\dline}}
  \put(374, 60){\usebox{\dline}}
  \put(390, 30){\framebox(150, 30){}}
  \multiput(420,30)(30,0){4}{\line(0,1){30}}
\end{picture}}
\put(  0, 70){\begin{picture}(200,50)
  \put(  0,  0){\shortstack[l]{Value of the\\SACLIB\\list handle: \BETA\ +}}
\end{picture}}
\put(175, 70){\begin{picture}(450,50)
  \put( 10,  0){0}
  \put( 40,  0){1}
  \put(100,  0){3}
  \put(160,  0){5}
  \multiput( 30,-45)( 60,  0){3}{\usebox{\cell}}
  \put(325,  0){\makebox(0,0)[b]{\NU-3}}
  \put(385,  0){\makebox(0,0)[b]{\NU-1}}
  \multiput(310,-45)( 60,  0){2}{\usebox{\cell}}
\end{picture}}
\end{picture}


\caption{The \SPACE\ array.}
\label{f:SPACE}
\end{figure}

The first \Word\ of each cell is used to store the handle of the next cell
in the list (i.e.\ the value returned by {\tt RED()}), while the second
\Word\ contains the data of the list element represented by the cell (i.e.\ 
the value returned by {\tt FIRST()}). Figure \ref{f:LIST} gives a graphical
representation of the cell structure for a sample list. The arrows stand
for list handles.

\begin{figure}[htb]
\unitlength0.5pt
\begin{picture}(700,80)
\savebox{\cell}(0,0)[b]{
  \multiput(-35, 15)(10,0){3}{\circle*{2}}
  \put(-15, 00){\usebox{\dline}}
  \put(-15, 30){\usebox{\dline}}
  \put(  0,  0){\framebox(80,30){}}
  \put( 40,  0){\line(0,1){ 30}}
  \put( 80, 00){\usebox{\dline}}
  \put( 80, 30){\usebox{\dline}}
  \multiput( 95, 15)(10,0){3}{\circle*{2}}
}
\savebox{\arrow}(0,0)[b]{
  \put(  0,  0){\circle*{4}}
  \put(  0,  0){\line(0,1){ 30}}
  \put(  0, 30){\line(1,0){110}}
  \put(110, 30){\vector(0,-1){ 15}}
}

\put( 40, 50){\begin{picture}(20,30)
  \put(  0, 20){\makebox(0,0)[b]{\ttL}}
  \put(  0, 15){\vector(0,-1){ 15}}
\end{picture}}
\put( 40, 20){\begin{picture}(600,100)
  \multiput(0,0)(130,  0){5}{\usebox{\cell}}
  \multiput(20,15)(130,  0){2}{\usebox{\arrow}}
  \put(410,15){\usebox{\arrow}}
  \put( 60,15){\makebox(0,0){1}}
  \put(280,15){\makebox(0,0){\NIL}}
  \put(320,15){\makebox(0,0){8}}
  \put(450,15){\makebox(0,0){9}}
  \put(540,15){\makebox(0,0){\NIL}}
  \put(580,15){\makebox(0,0){6}}
  \put(190, 15){\begin{picture}(200,30)
    \put(  0,  0){\circle*{4}}
    \put(  0,  0){\line(0,-1){ 30}}
    \put(  0,-30){\line(1,0){200}}
    \put(200,-30){\vector(0,1){ 15}}
  \end{picture}}
\end{picture}}
\end{picture}


\caption{The cell structure of the list $\ttL = (1,(9,6),8)$.}
\label{f:LIST}
\end{figure}

As already mentioned in Chapter \ref{c:LP}, atoms are required to be
integers $a$ with $-\BETA < a < \BETA$. This allows the garbage collector
and other functions operating on objects to decide whether a variable of
type \Word\ contains an atom or a list handle. Note that values less or
equal $-\BETA$ are legal only during garbage collection while values
greater than $\BETA + \NU$ are used for referencing other garbage collected
structures.

The \Word s of a cell adressed by a list handle \ttL\ are {\tt
\SPACE[$\ttL-\BETA$]} and {\tt \SPACE[$\ttL-\BETA+1$]}. To simplify these
computations, the C pointers \SPACEB\ and \SPACEBone\ are set to the memory
addresses of {\tt \SPACE[$-\BETA$]} and {\tt \SPACE[$-\BETA+1$]},
respectively. This is used by the functions {\tt FIRST(L)}, which returns
{\tt SPACEB1[L]}, and {\tt RED(L)}, which returns {\tt SPACEB[L]}.


%%%%%%%%%%%%%%%%%%%%%%%%%%%%%%%%%%%%%%%%%%%%%%%%%%%%%%%%%%%%%%%%%%%%%%%%%%%
\subsection{Implementation of GCA Handles}

When \saclib\ is initialised, the array \GCASPACE\ containing $\NUp+1$
structures of type \GCArray\ is allocated. A {\em GCA handle} is defined to be
\BETAp plus the index of the corresponding \GCArray\ structure in the
\GCASPACE\ array, with the null handle being \NIL.

The \GCArray\ structure contains the following fields:
\begin{itemize}
\item {\tt next}\ \ldots\
  a \Word, used for linking empty \GCArray s to the \GCAAVAIL\ list and for
  marking (see Section \ref{cNIWsGCssGC}).
\item {\tt flag}\ \ldots\
  a \Word, set to one of {\tt GC\_CHECK} and {\tt GC\_NO\_CHECK} (see Section
  \ref{cNIWsGCssGC}).
\item {\tt len}\ \ldots\
  a \Word, the length of the array in \Word s.
\item {\tt array}\ \ldots\
  a C pointer to an array of \Word s of size {\tt len}.
\end{itemize}

When {\tt GCAMALLOC()(} is called, it takes the first \GCArray\ from the
\GCAAVAIL\ list and initializes its fields.

{\tt GCA2PTR()} simply returns the C pointer in the {\tt array} field.

{\tt GCAFREE()} deallocates the memory addressed by the {\tt array} field,
sets all fields to zero, and links the \GCArray\ to the beginning of the 
\GCAAVAIL\ list. 


%%%%%%%%%%%%%%%%%%%%%%%%%%%%%%%%%%%%%%%%%%%%%%%%%%%%%%%%%%%%%%%%%%%%%%%%%%%
\subsection{The Garbage Collector}
\label{cNIWsGCssGC}

Garbage collection is invoked when {\tt COMP()} or {\tt GCAMALLOC()} call
{\tt GC()} in the case of \AVAIL\ or \GCAAVAIL\ being \NIL. The garbage
collector consists of two parts:
\begin{itemize}
\item
  The function {\tt GC()}\index{GC} is system dependent. It must ensure that
  the contents of all processor registers are pushed onto the stack and pass
  alignment information and the address of the end of the stack to
  {\tt GCSI()}.
\item
  {\tt GCSI()}\index{GCSI} is the system independent part of the garbage
  collector. It uses a mark-and-sweep method for identifying unused cells:
  \begin{description}
  \item[Marking:]
    The processor registers, the system stack, and the variables and GCA
    arrays to which pointers are stored in the \GCGLOBALS\ list are
    searched for non-\NIL\ list and GCA handles. All the cells accessible
    from these handles are marked by a call to {\tt MARK()}.

    If a list handle is found, this function traverses the cells of the
    list, marking them by negating the contents of their first \Word. If
    the second \Word\ of a cell contains a list or GCA handle, {\tt MARK()}
    calls itself recursively on this handle.

    In case of a GCA handle, the GCA cell adressed by the handle is marked
    by negating the contents of its {\tt next} field. If the cell's {\tt
    flag} field is not set to {\tt GC\_NO\_CHECK}, the \Word s in the array
    pointed to by the {\tt array} field are searched for list or GCA
    handles with {\tt MARK()} calling itself recursively on valid handles.
  \item[Sweeping:]
    In the sweep step, the \AVAIL\ and \GCAAVAIL\ lists are built:

    Cells in \SPACE\ whose first \Word\ contains a positive value are
    linked to the \AVAIL\ list. If the first \Word\ of a cell contains a
    negative value, it is made positive again and the cell is not changed
    in any other way.

    Cells in \GCASPACE\ whose {\tt next} field contains a positive value
    are linked to the \GCAAVAIL\ list and the array pointed to by the {\tt
    array} field is deallocated. If the {\tt next} field contains a
    negative value, it is made positive again and the cell is not changed
    in any other way.
  \end{description}
\end{itemize}

If the \AVAIL\ list contains no more than $\NU/\RHO$ cells at the end of
garbage collection, an error message is written to the output stream and
the program is aborted.

